\renewcommand{\chaptername}{Capítulo}
\chapter{Considerações finais}
\renewcommand{\chaptername}{Considerações finais}
\label{cap:consideracoes-finais}

Meus primeiros projetos de pesquisa (Seções \ref{sec:projeto-Daiana},
\ref{sec:projeto-Andre}, \ref{sec:projeto-Diego} e \ref{sec:projeto-Marcela}) envolveram 
temas diferentes daqueles que abordei ao longo do meu mestrado e doutorado e
foram fundamentais para consolidar a minha independência no ON.
A escolha dos temas foi feita em parte pelo perfil dos 
estudantes que orientei e em parte pelo tempo diminuto que é comumente concedido 
para a conclusão do mestrado com submissão de artigo científico. 
Com estes projetos aprendi uma lição que considero preciosa: o(a) pesquisador(a) 
submetido(a) a métricas pautadas no produtivismo acadêmico não pesquisa, necessariamente, 
aquilo que quer, mas sim o que é possível a partir dos recursos humanos, físicos e do
tempo que dispõe. Considero que os projetos desenvolvidos posteriormente mostram
uma pesquisa mais madura e consistente.

Em 2016, surgiu o grupo de \textbf{P}roblemas \textbf{In}versos 
em \textbf{G}eofísic\textbf{a}, o \href{https://www.pinga-lab.org/}{PINGA-lab}, 
que hoje é formado por mim, pela Valéria e por nossos estudantes de 
pós-graduação em geofísica do ON. O \href{https://www.pinga-lab.org/}{site do grupo} 
está em constante atualização e concentra toda a nossa produção científica. 
Desde a criação deste grupo temos incentivado a prática da ciência aberta de maneira 
mais sistemática e, sempre que possível, disponibilizamos todos os códigos desenvolvidos 
ao longo dos trabalhos na internet. A concepção deste grupo surgiu com meu amigo e 
parceiro de trabalho Leonardo Uieda, que hoje é professor na Universidade de Liverpool,
Inglaterra.

Ao longo dos nove anos em que venho atuando como pesquisador no ON, 
busquei me consolidar como um pesquisador independente, ainda que eu tenha trabalhado
e continue trabalhando com a Valéria, minha orientadora na pós-graduação.
Praticamente todos os trabalhos desenvolvidos nas dissertações e teses que 
orientei foram publicados em periódicos indexados que figuram entre 
os mais altos extratos do Qualis CAPES.
Todas as colaborações que estabeleci foram feitas com o intuito de levar aquilo que 
desenvolvemos em nosso grupo de pesquisa para trabalhar em conjunto com 
pesquisadores(as) de outras instituições. 
Ao longo destes anos, acho que consegui atuar não apenas como membro, mas também 
como coordenador, em conjunto com a Valéria, do nosso grupo de pesquisa 
no Observatório Nacional.
Considero que a minha atuação contribui para a pesquisa e formação de pesquisadores na 
área de geofísica, o que está em perfeita consonância com a missão institucional do
Observatório Nacional. 

Em termos de atividades institucionais, minha atuação no ON sempre foi predominantemente
concentrada na pós-graduação, seja como docente, orientador, membro da comissão de 
pós-graduação ou coordenador do programa.
O período em que atuei como coordenador da pós (2017-2018) foi complicado porque estava
orientando quatro estudantes de doutorado e uma estudante de mestrado simultaneamente. 
A coordenação impactou muito a condução destas orientações e teve como principal
consequência negativa a ausência de produção científica no meu currículo em 2018. 
O período 2017-2018 coincidiu com a avaliação quadrienal da CAPES.
Na ocasião, a comissão de avaliação considerou, de forma equivocada, que o nosso programa 
deveria permanecer com nota 4. Por conta disso, tive que fazer uma 
minuciosa revisão das informações contidas na Plataforma Sucupira, 
uma avaliação criteriosa dos principais pontos levantados pela comissão em nossa 
ficha de avaliação e, enfim, a elaboração de um pedido de reconsideração. 
Felizmente, a comissão de avaliação reconheceu as inconsistências e nosso programa 
recebeu a então merecida nota 5. Vale lembrar que, naquela época, fazia cerca de quatro anos 
e meio desde a minha posse como pesquisador no ON e cerca de cinco anos desde a defesa do 
meu doutorado.
Hoje, considero que aquela época foi um aprendizado sobre temas que, apesar de não
estarem minimamente relacionados à pesquisa, estão presentes na carreira de todo(a)
pesquisador(a). Acho que hoje estou mais preparado para atuar na coordenação do nosso
programa, se isso for necessário, e também em outras atividades institucionais.