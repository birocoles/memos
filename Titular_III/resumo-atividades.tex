\renewcommand{\chaptername}{Capítulo}
\chapter{Resumo das atividades científicas}
\renewcommand{\chaptername}{Resumo das atividades científicas}
\label{cap:resumo-atividades}

Neste capítulo, apresento um resumo das atividades científicas -- que incluem pesquisa, 
ensino, orientação e atividades institucionais -- desenvolvidas por mim ao longo da minha
carreira. É importante ressaltar que, no presente capítulo,
as atividades são apresentadas sem a devida contextualização, de forma meramente 
descritiva. Uma apresentação mais detalhada sobre as atividades de pesquisa, ensino e
orientação é feita no Capítulo \ref{cap:detalhe-atividades}.


\section{Pesquisa}
\label{sec:pesquisa}

Desde quando iniciei a minha carreira científica realizo pesquisa básica e aplicada em
Geofísica, com ênfase na área de métodos potenciais, sobre os seguintes temas:

\begin{itemize}
	
	\item[\parbox{0.03\textwidth}{\vspace{-0.1\baselineskip}\faSearch}] 
	{\textbf{Teoria do potencial aplicada:} investigações teóricas sobre transformações de campos gravitacionais e magnéticos produzidos por fontes 3D;}
	
	\item[\parbox{0.03\textwidth}{\vspace{-0.1\baselineskip}\faSearch}] 
	{\textbf{Métodos numéricos:} desenvolvimento de métodos computacionalmente eficientes para o processamento e interpretação de grandes volumes de dados;}
	
	\item[\parbox{0.03\textwidth}{\vspace{-0.1\baselineskip}\faSearch}] 
	{\textbf{Inversão de dados gravimétricos e magnéticos:} desenvolvimento de métodos para estimar a posição e a forma de fontes 3D;}
	
	\item[\parbox{0.03\textwidth}{\vspace{-0.1\baselineskip}\faSearch}] 
	{\textbf{Magnetização de corpos geológicos:} desenvolvimento de métodos para estimar a direção de magnetização de fontes 3D a partir de dados magnéticos provenientes de levantamentos aéreos e terrestres;}
	
	\item[\parbox{0.03\textwidth}{\vspace{-0.1\baselineskip}\faSearch}] 
	{\textbf{Magnetização de amostras de rocha:} desenvolvimento de métodos para estimar a direção de magnetização de amostras de rocha a partir de dados de microscopia magnética por varredura;}
	
	\item[\parbox{0.03\textwidth}{\vspace{-0.1\baselineskip}\faSearch}] 
	{\textbf{Modelagem gravimétrica e magnética:} desenvolvimento de métodos e implementação computacional de algoritmos para calcular os campos gravitacional e magnético produzidos por fontes 3D e o cálculo do campo desmagnetizante no interior de corpos geológicos com alta suscetibilidade;}
	
	\item[\parbox{0.03\textwidth}{\vspace{-0.1\baselineskip}\faSearch}] 
	{\textbf{Caracterização regional do campo de gravidade:} desenvolvimento de métodos para a representação do campo de gravidade regional a partir da combinação de diferentes conjuntos de dados gravimétricos;}
	
	\item[\parbox{0.03\textwidth}{\vspace{-0.1\baselineskip}\faSearch}] 
	{\textbf{Caracterização regional do campo magnético crustal:} desenvolvimento de métodos para a representação do campo magnético crustal a partir da combinação de diferentes conjuntos de dados magnéticos.}
	
\end{itemize}

\section{Observatório Nacional, Brasil}

\subsection*{Vínculo institucional}

\begin{itemize}
	\item[\parbox{0.03\textwidth}{\vspace{-0.3\baselineskip}\faClipboardList}]
	{Pesquisador Titular II \dotfill \parbox{0.11\textwidth}{\hfill 2021--hoje}} 
	\item[\parbox{0.03\textwidth}{\vspace{-0.3\baselineskip}\faClipboardCheck}]
	{Pesquisador Titular I \dotfill \parbox{0.11\textwidth}{\hfill 2020--2021}}
	\item[\parbox{0.03\textwidth}{\vspace{-0.3\baselineskip}\faClipboardCheck}]
	{Pesquisador Associado III \dotfill \parbox{0.11\textwidth}{\hfill 2018--2020}}
	\item[\parbox{0.03\textwidth}{\vspace{-0.3\baselineskip}\faClipboardCheck}]
	{Pesquisador Associado II \dotfill \parbox{0.11\textwidth}{\hfill 2017--2018}}
	\item[\parbox{0.03\textwidth}{\vspace{-0.3\baselineskip}\faClipboardCheck}]
	{Pesquisador Associado I \dotfill \parbox{0.11\textwidth}{\hfill 2016--2017}}
	\item[\parbox{0.03\textwidth}{\vspace{-0.3\baselineskip}\faClipboardCheck}]
	{Pesquisador Adjunto III \dotfill \parbox{0.11\textwidth}{\hfill 2015--2016}}
	\item[\parbox{0.03\textwidth}{\vspace{-0.3\baselineskip}\faClipboardCheck}]
	{Pesquisador Adjunto II \dotfill \parbox{0.11\textwidth}{\hfill 2014--2015}}
	\item[\parbox{0.03\textwidth}{\vspace{-0.3\baselineskip}\faClipboardCheck}]
	{Pesquisador Adjunto I \dotfill \parbox{0.11\textwidth}{\hfill 2013--2014}}
	
\end{itemize}

\subsection*{Atividades institucionais}

\begin{itemize}
	
	\item[\parbox{0.03\textwidth}{\vspace{-0.1\baselineskip}\faUserCheck}]
	{Membro da comissão de pós-graduação em Geofísica \dotfill \parbox{0.11\textwidth}{\hfill 2020--hoje} \newline \textit{Portaria ON 20/2020 - 5/08/2020}} 
	
	\item[\parbox{0.03\textwidth}{\vspace{-0.1\baselineskip}\faUserCheck}]
	{Coordenador do programa de pós-graduação em Geofísica \dotfill \parbox{0.11\textwidth}{\hfill 2017--2018} \newline \textit{Portaria ON 22/2017 - 29/05/2017}}
	
	\item[\parbox{0.03\textwidth}{\vspace{-0.1\baselineskip}\faUserCheck}]
	{Membro da comissão de dados abertos \dotfill \parbox{0.11\textwidth}{\hfill 2017--2019} \newline \textit{Portaria ON 7/2017 - 8/03/2017}}

	\item[\parbox{0.03\textwidth}{\vspace{-0.1\baselineskip}\faUserCheck}]
	{Membro da comissão de pós-graduação em Geofísica \dotfill \parbox{0.11\textwidth}{\hfill 2015--2017} \newline \textit{Portaria ON 7/2015 - 18/03/2015}}

	\item[\parbox{0.03\textwidth}{\vspace{-0.1\baselineskip}\faUserCheck}]
	{Membro do corpo docente permanente do programa de pós-graduação em Geofísica \dotfill \parbox{0.11\textwidth}{\hfill 2014--hoje} \newline \textit{Portaria ON 7/2015 - 18/03/2015}}

	\item[\parbox{0.03\textwidth}{\vspace{-0.1\baselineskip}\faUserCheck}]
	{Membro da comissão do Programa de Capacitação Institucional (PCI) \dotfill \parbox{0.11\textwidth}{\hfill 2013--2014} \newline \textit{Portaria ON 44/2013 - 15/10/2013}}
	
\end{itemize}


\subsection*{Atividades de ensino e orientação}


Ingressei como membro do corpo docente permanente do Programa de Pós-Graduação em
Geofísica do Observatório Nacional (PPGG-ON) em 2014. Desde então, 
concluí a orientação de \textbf{4 (quatro)} teses de doutorado e 
\textbf{4 (quatro)} dissertações de mestrado como \textbf{orientador principal},
bem como \textbf{2 (duas)} teses de doutorado e \textbf{2 (duas)} dissertações de
mestrado como \textbf{co-orientador} sobre diversos temas relacionados às minhas
linhas de pesquisa. Atualmente, oriento \textbf{2 (duas)} dissertações de mestrado
como orientador principal e \textbf{2 (duas)} teses de doutorado como co-orientador
no PPGG-ON. 
Além disso, atuo como docente responsável desde 2014 por \textbf{2 (duas)} disciplinas 
no PPGG-ON.


\section{Institut Universitaire Européen de la Meer (IUEM), França}


\subsection*{Vínculo institucional}


\begin{itemize}
	
	\item[\parbox{0.03\textwidth}{\vspace{-0.3\baselineskip}\faClipboardCheck}]
	{Pesquisador visitante \dotfill \parbox{0.11\textwidth}{\hfill 2018--2019}} 
	
\end{itemize}

\subsection*{Atividades de ensino e orientação}

Ao longo dos anos de 2018 e 2019, atuei como pesquisador visitante no 
\href{https://www-iuem.univ-brest.fr/lgo/le-labo/}{\textit{Laboratoire Géosciences Océan}}, 
do
\href{https://www-iuem.univ-brest.fr/}{\textit{Institut Universitaire Européen de la Meer}},
França.
Naquele período, trabalhei juntamente com a 
\href{https://www-iuem.univ-brest.fr/lgo/equipe/maia-marcia-do-carmo/}{Dra. Marcia Maia} 
e o 
\href{https://www-iuem.univ-brest.fr/lgo/equipe/tarits-pascal/}{Dr. Pascal Tarits} 
na interpretação de dados magnéticos na Zona de Falhas Transformantes de São
Paulo (ZFTSP), próximo ao arquipélago de São Pedro e São Paulo, e
\textbf{co-orientei um trabalho de mestrado}.



\section{Liverpool University, Reino Unido}


\subsection*{Vínculo institucional}


\begin{itemize}
	
	\item[\parbox{0.03\textwidth}{\vspace{-0.3\baselineskip}\faClipboardList}]
	{Honorary PGR Supervisor \dotfill \parbox{0.11\textwidth}{\hfill 2021--2024}} 
	
\end{itemize}

\subsection*{Atividades de ensino e orientação}

Desde 2021, atuo como \textbf{co-orientador de uma tese de doutorado} no 
âmbito do projeto de pesquisa ``\textit{Improving estimates of Antarctic geothermal heat flow from magnetic data}". Este projeto está sendo desenvolvido no departamento de
\href{https://www.liverpool.ac.uk/earth-ocean-and-ecological-sciences/}{\textit{Earth, Ocean and Ecological Sciences}} da 
\href{https://www.liverpool.ac.uk/}{\textit{Liverpoool University}}, Reino Unido, em
conjunto com o 
\href{https://www.liverpool.ac.uk/environmental-sciences/staff/leonardo-uieda/}{Dr. Leonardo Uieda} (orientador principal e coordenador do projeto) e o
\href{https://www.liverpool.ac.uk/environmental-sciences/staff/richard-holme/}{Dr. Richard Holme}.


\section{Compilação das principais atividades}


Abaixo apresento uma lista com as minhas principais atividades científicas:

\begin{itemize}
	\item[\parbox{0.03\textwidth}{\vspace{-0.1\baselineskip}\faSeedling}]
	\textbf{2 (duas)} orientações de mestrado em andamento como orientador principal;
	\item[\parbox{0.03\textwidth}{\vspace{-0.1\baselineskip}\faSeedling}]
	\textbf{2 (duas)} orientações de doutorado em andamento como coorientador;
	\item[\parbox{0.03\textwidth}{\vspace{-0.1\baselineskip}\faSeedling}]
	\textbf{1 (uma)} supervisão de pós doutorado em andamento;
	\item[\parbox{0.03\textwidth}{\vspace{-0.1\baselineskip}\faBaby}]
	Conclusão de \textbf{4 (quatro)} orientações de mestrado como orientador principal e outras \textbf{3 (três)} como co-orientador;
	\item[\parbox{0.03\textwidth}{\vspace{-0.1\baselineskip}\faChild}]
	Conclusão de \textbf{4 (quatro)} orientações de doutorado como orientador principal e outras \textbf{2 (duas)} como co-orientador;
	\item[\parbox{0.025\textwidth}{\vspace{-0.1\baselineskip}\faMale}]
	Conclusão de \textbf{2 (duas)} supervisões de pós doutorado;
	\item[\parbox{0.03\textwidth}{\vspace{-0.1\baselineskip}\faPencil*}]
	Publicação de \textbf{20 (vinte)} artigos em periódicos de circulação internacional;
	\item[\parbox{0.03\textwidth}{\vspace{-0.1\baselineskip}\faPeopleCarry}]
	Desenvolvimento de cooperações nacionais e internacionais;
	\item[\parbox{0.03\textwidth}{\vspace{-0.1\baselineskip}\aiIDEASRePEc}] 
	Coordenação de \textbf{2 (dois)} projetos financiados pelo CNPq
	(\textit{Universal} e \textit{Bolsa PQ});
	\item[\parbox{0.03\textwidth}{\vspace{-0.1\baselineskip}\aiIDEASRePEc}]
	Coordenação de \textbf{2 (dois)} projetos financiados pela FAPERJ
	(\textit{Auxílio Instalação} e \textit{JCNE});
	\item[\parbox{0.03\textwidth}{\vspace{-0.1\baselineskip}\faBomb}]
	Coordenação do Programa de Pós-Graduação em Geofísica do ON.

\end{itemize}
