\section*{Período antes de Jul/2013}


Trabalho com desenvolvimento de métodos numéricos para o processamento e
interpretação de dados gravimétricos e magnetométricos desde quando comecei
a graduação em geofísica, em 2004, no IAG-USP, São Paulo.
Desde aquela época eu queria ser cientista, ainda que este conceito tenha mudado 
bastante para mim ao longo dos anos. Se por um lado essa certeza me ajudou a 
ter muito foco nas minhas decisões, por outro lado ela me privou de experimentar
o caminho da iniciativa privada. Mas acho que foi melhor assim.
O tema de pesquisa que escolhi me conduziu ao Rio de Janeiro, quando iniciei 
o mestrado em geofísica no Observatório Nacional (ON), em Março de 2009, sob 
a orientação da pesquisadora Dra. Valéria C. F. Barbosa. 
Terminei o mestrado em pouco menos de dois anos e iniciei o doutorado no ON, 
também sob a orientação da Dra Valéria Barbosa, em Dezembro de 2010.

Mais importante do que o conhecimento técnico acumulado na graduação e, 
principalmente, no mestrado e doutorado, acho que a independência acadêmica 
foi a principal lição que aprendi com a minha orientadora naquela época.
Obviamente, isso não significa que eu estava preparado para seguir a minha 
carreira completamente sozinho desde aquela época. O que quero dizer com 
independência acadêmica é que eu tinha começado a ter minhas próprias ideias,
ainda que fossem ruins. Eu passei a conseguir investiga-las de forma 
relativamente independente, comecei a perceber o que eu precisava estudar 
mais e quais eram as minhas principais limitações.
No doutorado aprendi que o pesquisador deve ser um especialista em aprender
e que o rigor científico necessário para chegar a uma determinada conclusão é 
mais importante do que a própria conclusão.
Divagações à parte, houve um concurso para pesquisador do ON na área de métodos 
potenciais, no final de 2012, no meio do meu doutorado. Fui aprovado no concurso
e, por conta disso, defendi meu doutorado em Janeiro de 2013, aproximadamente dois 
anos após ter começado.


\section*{Período após Jul/2013}


Minha carreira de pesquisador no ON começou com a co-orientação 
de uma dissertação de mestrado. No ano seguinte, em 2014, assumi como orientador 
principal e a Dra. Valéria Barbosa como co-orientadora. 
O projeto de pesquisa visava desenvolver um método para estimar 
a direção de magnetização total de corpos geológicos que pudessem ser aproximados
por esferas. A partir daquela ideia relativamente simples, pude conduzir a orientação da 
dissertação e tive meus primeiros projetos de pesquisa aprovados por agências de fomento 
(subseção \ref{projeto-Daiana}). Aquele projeto foi muito importante para eu começar a 
orientar e elaborar meus primeiros projetos de pesquisa. Até hoje penso que orientar 
é a atividade mais difícil na carreira de um(a) pesquisador(a).

Ainda em 2014, comecei a orientar outra dissertação de mestrado, desta vez sem a 
co-orientação da Dra. Valéria. O projeto propõe o desenvolvimento de métodos 
para o processamento e interpretação de dados de microscopia magnética de varredura e
marcou o início da colaboração que mantenho com pesquisadores PUC-RIO até hoje
(subseção \ref{projeto-Andre}). 
Ainda que o projeto seja basicamente a aplicação e técnicas comumente usadas na 
magnetometria para interpretar dados produzidos por amostras de rocha, ele representou 
uma ligeira mudança de área na minha carreira. Tão importante quanto concluir a 
orientação de outra dissertação de mestrado e estabelecer minha primeira colaboração foi 
publicar, ao longo deste projeto, meus primeiros trabalhos sem a participação da 
minha amiga e principal parceira de trabalho, Dra. Valéria. Isso foi fundamental para 
eu poder seguir minha carreira dentro do ON. Obviamente voltei a 
trabalhar com ela em projetos posteriores, mas dessa vez por opção e não por necessidade.
Outros dois projetos iniciados em 2015 (subseção \ref{projeto-Diego}) e 2016 
(subseção \ref{projeto-Marcela}), ambos concluídos, também foram importantes para 
consolidar minha independência na carreira.

Os projetos de pesquisa que desenvolvi até 2016 envolvem temas que considero 
relativamente simples e que são diferentes daqueles que abordei ao longo do meu 
mestrado e doutorado. A escolha destes temas foi feita em parte pelo perfil dos 
estudantes que orientei e em parte pelo tempo diminuto que é comumente concedido 
para a conclusão do mestrado com submissão de artigo científico. 
Com estes projetos aprendi uma lição que considero preciosa: o(a) pesquisador(a) 
submetido(a) a métricas pautadas no produtivismo acadêmico não pesquisa, necessariamente, 
aquilo que quer, mas sim o que é possível a partir dos recursos humanos, físicos e do
tempo que dispõe.

Em 2016, surgiu o grupo de \textbf{P}roblemas \textbf{In}versos 
em \textbf{G}eofísic\textbf{a}, o \href{https://www.pinga-lab.org/}{PINGA-lab}, 
que hoje é formado por mim, pela Dra. Valéria Barbosa e por nossos estudantes de 
pós-graduação em geofísica do ON. O \href{https://www.pinga-lab.org/}{site do grupo} 
está em constante atualização e concentra toda a nossa produção científica. 
Desde a criação deste grupo temos incentivado a prática da ciência aberta de maneira 
mais sistemática e, sempre que possível, disponibilizamos todos os códigos desenvolvidos 
ao longo dos trabalhos na internet. A concepção deste grupo surgiu com meu amigo e 
parceiro de trabalho \href{https://www.leouieda.com/}{Dr. Leonardo Uieda}, que hoje 
é pesquisador na Universidade de Liverpool, Inglaterra, e também concluiu seu mestrado 
e doutorado no ON, sob orientação da Dra. Valéria Barbosa.

Ainda em 2016, concluí a orientação de uma tese de doutorado na condição de 
co-orientador, em conjunto com Dra. Valéria Barbosa. Também iniciei a orientação 
dos meus primeiros estudantes de doutorado como orientador principal e iniciei um 
projeto de pesquisa que visa o desenvolvimento de métodos computacionalmente eficientes 
para o processamento e interpretação de dados de campo potenciais (subseção 
\ref{projeto-guarda-chuva-fast}). Este projeto guarda certa relação com um tema que 
abordei em meu doutorado, encontra-se em andamento e possibilitou o 
desenvolvimento de métodos iterativos que reduziram drasticamente o custo computacional
da técnica da camada equivalente aplicada a dados gravimétricos. 
Ao longo deste projeto foi possível mostrar, pela primeira vez, 
como a camada equivalente pode ser formulada em termos de uma convolução discreta e,
consequentemente, resolvida de forma extremamente eficiente usando a Transformada 
Rápida de Fourier. 
Outro ponto que destaco neste projeto é o começo de uma colaboração com o pesquisador 
Dr. Maurizio Fedi, da \textit{Università Degli Studi Di Napoli Federico II}, na Itália.

Outro projeto que iniciei em 2016 (subseção \ref{projeto-Leo}) visa o desenvolvimento 
de métodos para a inversão de anomalias magnéticas com o intuito de estimar a geometria 
de corpos geológicos 3D. Este projeto representa uma continuação de temas que abordei 
em meu mestrado e doutorado. A partir deste projeto, foi possível estabelecer uma 
colaboração com o Dr. Clive Foss, do \textit{Commonwealth Scientific and Industrial 
Research Organisation} (CSIRO), Austrália, que é especialista em geofísica de exploração. 
O primeiro artigo relacionado a este projeto, \textit{Magnetic radial inversion for 3-D 
source geometry estimation}, é fruto da tese de doutorado do estudante Leonardo Vital 
e foi submetido ao periódico \textit{Geophysical Journal International}. O artigo está 
passando pela sua primeira rodada de revisões e recebeu parecer favorável a publicação.

O período entre Março de 2017 a Março de 2018 foi extremamente complicado para mim,
uma vez que tive que assumir a coordenação do programa de pós-graduação em geofísica 
do ON. Na época eu estava orientando quatro estudantes de doutorado e uma estudante de 
mestrado simultaneamente. A coordenação impactou muito a condução destas orientações 
e teve como principal consequência negativa a ausência de produção científica no meu 
currículo em 2018. O período 2017-2018 coincidiu com a avaliação quadrienal da CAPES.
Na ocasião, a comissão de avaliação considerou, de forma equivocada, que o nosso programa 
deveria permanecer com nota 4. Por conta disso, tive que coordenar um processo 
extremamente desgastante que envolveu várias reuniões com o corpo docente do nosso 
programa, uma minuciosa revisão das informações contidas na Plataforma Sucupira, 
uma avaliação criteriosa dos principais pontos levantados pela comissão em nossa 
ficha de avaliação e, enfim, a elaboração de um pedido de reconsideração. 
Felizmente, a comissão de avaliação reconheceu as inconsistências e nosso programa 
recebeu a então merecida nota 5. Vale lembrar que, naquela época, fazia cerca de 4 anos 
e meio desde a minha posse como pesquisador no ON e cerca de 5 anos desde a defesa do 
meu doutorado. Os impactos negativos sobre a minha carreira, sobretudo no ano de 2018,
fizeram com que eu não atingisse, pela primeira vez até então, a pontuação necessária 
para progredir na carreira. 
Hoje, considero que aquela época foi um aprendizado sobre temas que, apesar de não
estarem minimamente relacionados à pesquisa, estão presentes na carreira de todo(a)
pesquisador(a).

Mas nem tudo foi ruim naquela época. Em 2017, iniciei um projeto de pesquisa para 
desenvolver métodos de processamento e interpretação de dados magnéticos a partir 
da técnica da camada equivalente (subseção \ref{projeto-guarda-chuva-eqlayer}). 
Este projeto foi aprovado para receber financiado do CNPq via bolsa PQ e é a 
principal linha de pesquisa do nosso grupo atualmente. 
Em um trabalho recente desenvolvido no âmbito deste projeto mostramos, pela primeira
vez, que a distribuição de intensidade de momento magnético sobre uma camada plana 
de dipolos é toda positiva quando esta possui a mesma direção de magnetização uniforme 
das fontes magnéticas verdadeiras. Este resultado teórico possibilitou o desenvolvimento 
de um método para estimar a direção de magnetização total de fontes com direção de 
magnetização uniforme, sem a necessidade de impor vínculos sobre a posição ou forma 
das fontes. O então estudante de doutorado André L. A. Reis, que desenvolveu esta 
pesquisa sob minha orientação, foi aprovado em concurso público em 2019 
para pesquisador no departamento de geologia da Universidade do Estado do Rio de 
Janeiro (UERJ) e defendeu sua tese em 2020, o que pra mim representa motivo de muito 
orgulho.

Em 2018, iniciei um projeto que visa o desenvolvimento de métodos para a interpretação 
de feições magnéticas crustais em regiões próximas ao equador e é financiado pela 
FAPERJ via bolsa JCNE (subseção \ref{projeto-baixas-latitudes}). 
No âmbito deste projeto, realizei visitas ao \textit{Institut Universitaire Européen 
de la Meer} (IUEM), França, ao longo de 2018 e 2019 para trabalhar em conjunto com 
os pesquisadores Dra. Marcia Maia e Dr. Pascal Tarits e co-orientar o trabalho de 
uma estudante de mestrado naquela instituição. 
Continuei esta pesquisa quando voltei da França em 2019 e recentemente iniciei 
a orientação do estudante de mestrado Raimundo Sousa Jr. neste tema, no programa
de pós-graduação em geofísica do ON.
Ainda em 2020, iniciei outro projeto de pesquisa que visa generalizar a redução ao 
polo para fontes que possuem distribuição de magnetização heterogênea (subseção 
\ref{projeto-mag-heterogenea}). Pesquisas sobre este tema estão sendo conduzidas 
com nossos estudantes de mestrado e doutorado para interpretar dados de microscopia 
magnética sobre amostras de rocha, anomalias magnéticas sobre depósitos minerais e 
também sobre anomalias magnéticas produzidas por fontes profundas na crosta de Marte
usando dados de satélite.

Ao longo dos pouco mais de sete anos em que venho atuando como pesquisador no ON, 
busquei me consolidar como um pesquisador independente. 
Praticamente todos os trabalhos desenvolvidos nas dissertações e teses que 
orientei foram publicados em periódicos indexados que figuram entre 
os mais altos extratos do Qualis CAPES.
Todas as colaborações que estabeleci foram feitas com o intuito de levar aquilo que 
desenvolvemos em nosso grupo de pesquisa para trabalhar em conjunto com 
pesquisadores(as) de outras instituições. 
Ao longo destes anos, acho que consegui atuar não apenas como membro, mas também 
como coordenador, em conjunto com a Dra. Valéria Barbosa, do nosso grupo de pesquisa 
no Observatório Nacional.
Considero que a minha principal linha de pesquisa hoje é o estudo teórico e o
desenvolvimento de métodos computacionalmente eficientes para processamento e 
interpretação de dados de campos potenciais, uma vez que isso define praticamente 
todos os projetos de pesquisa que desenvolvo atualmente. No momento, oriento dois
estudantes de doutorado, um com defesa prevista para 2020 e o outro para 2021, 
e dois estudantes de mestrado, ambos com defesa prevista para 2022. Também 
co-oriento duas estudantes de doutorado e um estudantes de mestrado.

