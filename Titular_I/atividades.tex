Neste capítulo, apresento as atividades que venho desenvolvendo desde 2011, 
em especial aquelas desenvolvidas a partir de Julho de 2013, quando tomei 
posse como pesquisador no Observatório Nacional. Estas atividades incluem:
\begin{itemize}
	%\setlength\itemsep{-0.1cm}
	\item Publicação de \textbf{16} artigos em periódicos de circulação
	internacional (um está ``aceito para publicação");
	\item Conclusão de \textbf{quatro} orientações de mestrado como orientador
	principal e outras \textbf{duas} como co-orientador;
	\item Conclusão de \textbf{duas} orientações de doutorado como orientador 
	principal e outras \textbf{duas} como co-orientador;
	\item Supervisão de \textbf{dois} pós doutorandos;
	\item Coordenação de \textbf{dois} projetos financiados pelo CNPq
	(\textit{Universal} e \textit{Bolsa PQ});
	\item Coordenação de \textbf{dois} projetos financiados pela FAPERJ
	(\textit{Auxílio Instalação} e \textit{JCNE});
	\item Desenvolvimento de cooperações nacionais e internacionais;
	\item Coordenação do Programa de Pós-Graduação em Geofísica do ON.
\end{itemize}



\section{Projetos de pesquisa}


%%%%%%%%%%% Projeto da Daiana
\subsection{Estimativa da direção da magnetização total de corpos 3D aproximadamente esféricos (2014 -- 2017)} \label{projeto-Daiana}

\subsubsection{\emph{Objetivos}}

Desenvolver uma metodologia computacionalmente eficiente para estimar a direção da 
magnetização total de corpos aproximadamente esféricos a partir da inversão da 
anomalia de campo total. Os recursos financeiros recebidos pelas agências de fomento 
foram utilizados para comprar material de consumo, prover recursos computacionais e 
custear a divulgação dos resultados em congressos internacionais.

\subsubsection{\emph{Formação de recursos humanos}}

\begin{enumerate}
	
	\item\ThesisEntry{Mestrado}{\href{http://www.pinga-lab.org/thesis/daiana-msc.html}{Estimativa do vetor de magnetiza\c{c}\~{a}o total de corpos aproximadamente esf\'{e}ricos}}{Daiana P. Sales}{Observat\'{o}rio Nacional}{2014}{Orientador principal}
	
\end{enumerate}

\subsubsection{\emph{Financiamentos}}

\begin{enumerate}
	
	\item\FundingEntry{\href{http://cnpq.br/pagina-inicial}{Conselho Nacional de Desenvolvimento Cient\'{i}fico e Tecnol\'{o}gico (CNPq)}}{Estimativa da dire\c{c}\~{a}o da magnetiza\c{c}\~{a}o total de corpos 3D aproximadamente esf\'{e}ricos}{445752/2014-9}{MCTI/CNPQ/Universal 14/2014}{20~000.00}{Nov/2014 -- Nov/2017}
	
	\item\FundingEntry{\href{http://www.faperj.br/}{Funda\c{c}\~{a}o Carlos Chagas Filho de Amparo \`{a} Pesquisa do Estado do Rio de Janeiro (FAPERJ)}}{Infraestrutura computacional para a estima\c{c}\~{a}o da magnetiza\c{c}\~{a}o de corpos 3D aproximadamente dipolares}{E-26/111.152/2014}{INST - Aux{\' i}lio Instala{\c c}{\~ a}o - 2013/2 }{10~000,00}{Jun/2014 -- Mar/2016}
	
\end{enumerate}

\subsubsection{\emph{Resumos em anais de congressos}}

\begin{enumerate}
	\item \bibentry{iugg_2015a}
\end{enumerate}

\subsubsection{\emph{Artigos em periódicos indexados}}

\begin{enumerate}
	\item \bibentry{oliveirajr_etal_2015}
\end{enumerate}


%%%%%%%%%%% Projeto do André - mestrado
\subsection{Desenvolvimento de métodos para processamento e interpretação de dados de microscopia magnética (2014 -- presente)} \label{projeto-Andre}

\subsubsection{\emph{Objetivos}}

Desenvolver métodos de processamento e interpretação de dados de microscopia magnética 
de varredura aplicada à caracterização de amostras de rocha para estudos paleomagnéticos 
e de magnetismo de rochas. Este projeto é desenvolvido em colaboração com 
pesquisadores do \href{https://www.fis.puc-rio.br/instrumentacao-e-medidas-magneticas/}{Laboratório de Instrumentação e Medidas Magnéticas da PUC-RIO}.

\subsubsection{\emph{Formação de recursos humanos}}

\begin{enumerate}
	
	\item\ThesisEntry{Mestrado}{\href{http://www.pinga-lab.org/thesis/andre-msc.html}{Invers\~{a}o magn\'{e}tica 3D para estimar a distribui\c{c}\~{a}o de magnetiza\c{c}\~{a}o de uma amostra de rocha}}{Andr{\'e} L. A. Reis}{Observat{\'o}rio Nacional, Brazil}{2016}{Orientador principal}
	
\end{enumerate}

\subsubsection{\emph{Resumos em anais de congressos}}

\begin{enumerate}
	\item \bibentry{latinmag_2017}
	\item \bibentry{agu_fallmeeting_2016a}
	\item \bibentry{iugg_2015b}
\end{enumerate}

\subsubsection{\emph{Artigos em periódicos indexados}}

\begin{enumerate}
	\item \bibentry{araujo_etal2019_materials}
	\item \bibentry{araujo_etal2019_sensors}
	\item \bibentry{reis_etal2016}
	
\end{enumerate}


%%%%%%%%%%% Projeto do Diego - mestrado
\subsection{Modelagem magnética de corpos com alta suscetibilidade (2015 -- 2017)} \label{projeto-Diego}

\subsubsection{\emph{Objetivos}}

É comum considerar que corpos geológicos são magnetizados uniformemente quando são 
submetidos a um campo magnético uniforme. Esta premissa falha quando os corpos possuem 
alta suscetibilidade magnética devido a um fenômeno chamado auto-desmagnetização 
(self-demagnetization). Negligenciar este efeito para corpos com alta suscetibilidade 
pode levar a resultados espúrios obtidos a partir da interpretação de anomalias 
magnéticas. 
Ao longo deste projeto, foi desenvolvido um método para a modelagem e interpretação 
de dados magnéticos produzidos por corpos geológicos que possuem altos valores de 
suscetibilidade e/ou anisotropia de suscetibilidade e que podem ser aproximados por 
elipsoides. Também foi desenvolvido um novo critério numérico para definir o limite 
no valor da suscetibilidade magnética isotrópica a partir do qual o efeito de 
auto-desmagnetização deve ser levado em consideração.

A dissertação desenvolvida ao longo deste projeto pelo estudante Diego Takahashi, sob 
minha orientação, foi finalista no \textit{Earth Model Award 2017}, organizado pela 
empresa Halliburton-Landmark.

\subsubsection{\emph{Formação de recursos humanos}}

\begin{enumerate}
	
	\item\ThesisEntry{Mestrado}{\href{http://www.pinga-lab.org/thesis/takahashi-msc.html}{Modelagem magn\'{e}tica 3D de corpos elipsoidais}}{Diego Takahashi}{Observat\'{o}rio Nacional, Brazil}{2017}{Orientador principal}
	
\end{enumerate}

\subsubsection{\emph{Resumos em anais de congressos}}

\begin{enumerate}
	\item \bibentry{sbgf_2016_elipsoid}
\end{enumerate}

\subsubsection{\emph{Artigos em periódicos indexados}}

\begin{enumerate}
	\item \bibentry{takahashi_oliveirajr2017}
	
\end{enumerate}


%%%%%%%%%%% Projeto da Marcela - mestrado
\subsection{Inversão de dados de campos potenciais para estimar a geometria de multiplas superfícies (2016 -- 2019)} \label{projeto-Marcela}

\subsubsection{\emph{Objetivos}}

Dados gravimétricos e magnetométricos são utilizados há muito tempo para investigar 
a estrutura crustal em estudos locais. A estratégia mais comum é aproximar a crosta por 
um conjunto de camadas justapostas com propriedade física constante ou dependente 
da profundidade. Os limites destas camadas são definidos por superfícies que geralmente 
representam camadas sedimentares, a batimetria, o relevo do embasamento e a Moho. 
Problemas inversos que visam estimar a geometria destas superfícies sofrem, geralmente, 
de uma severa falta de unicidade, sobretudo nas situações em que a geometria de duas 
ou mais superfícies devem ser estimadas simultaneamente. Ao longo deste projeto, 
foi desenvolvido um método para estimar a geometria do embasamento e da Moho sob  
uma margem continental passiva por meio da inversão não-linear de dados 
gravimétricos. Para contornar a ambiguidade inerente a este problema, um vínculo 
isostático foi desenvolvido para impor suavidade no stress litostático produzido pela 
crosta e manto superior sobre uma superfície plana com profundidade constante. 
O método se mostrou útil na interpretação de dados de satélite 
sobre uma margem vulcânica com pronunciado afinamento crustal no sul do Brasil. 
Outro aspecto relevante do projeto foi mostrar que a imposição de equilíbrio isostático 
não remove a ambiguidade do problema. De fato, mostrou-se que é possível obter modelos 
que ajustam os dados igualmente e com diferentes graus de equilíbrio isostático.


\subsubsection{\emph{Formação de recursos humanos}}

\begin{enumerate}
	
	\item\ThesisEntry{MSc}{\href{http://www.pinga-lab.org/thesis/marcela-msc.html}{Invers{\~ a}o gravim{\' e}trica 2D com v{\' i}nculo isost{\' a}tico}}{Barbara Marcela S. Bastos}{Observat\'{o}rio Nacional, Brazil}{2018}{Orientador principal}
	
	\item\ThesisEntry{MSc}{\href{http://www.pinga-lab.org/thesis/victor-msc.html}{Investiga\c{c}\~{a}o geof\'{i}sica do Alto do Cear\'{a} na margem equatorial brasileira -- Uma crosta continental ou uma crosta oce\^{a}nica?}}{Victor C. Pereira}{Observat\'{o}rio Nacional, Brazil}{2017}{Co-orientador}
	
\end{enumerate}

\subsubsection{\emph{Artigos em periódicos indexados}}

\begin{enumerate}
	\item \bibentry{bastos_oliveirajr2020}
	
\end{enumerate}


%%%%%%%%%%% Projeto guarda-chuva
\subsection{Métodos computacionalmente eficientes para o processamento, modelagem e interpretação de dados de campos potenciais (2016 -- presente)} \label{projeto-guarda-chuva-fast}

\subsubsection{\emph{Objetivos}}

Grandes volumes de dados gravimétricos e magnetométricos requerem a utilização de 
metodologias computacionalmente eficientes para a seu processamento e interpretação. 
Sob determinadas condições, os métodos numéricos utilizados para o processamento e 
interpretação de dados de campos potenciais envolvem sistemas lineares que podem 
ser resolvidos de forma eficiente, por meio de métodos iterativos. Em outras situações, 
estes sistemas são equivalentes à uma operação de convolução discreta. Esta operação, 
por sua vez, pode ser muito eficiente do ponto de vista computacional quando calculada 
usando a transformada rápida de Fourier. O presente projeto propõe o desenvolvimento 
de métodos computacionalmente eficientes para resolver problemas de  processamento, 
modelagem e interpretação de dados de campos potenciais que podem ser resolvidos de 
forma iterativa e/ou formulados como uma convolução discreta. 

Parte deste trabalho é uma colaboração com o Dr. Maurizio Fedi, da 
\textit{Università Degli Studi Di Napoli Federico II}, na cidade de Napoli, Itália, 
que é especialista em geofísica aplicada. O estudante de doutorado Diego Takahashi 
realizou estágio, sob a supervisão do Dr. Maurizio Fedi, no âmbito do edital 
N$^{\circ}$ 41/2018 PROGRAMA INSTITUCIONAL DE DOUTORADO SANDUÍCHE NO EXTERIOR 2018/2019
da CAPES, entre os anos de 2019 e 2020.

\subsubsection{\emph{Formação de recursos humanos}}

\begin{enumerate}
	
	\item\ThesisEntry{PhD}{M{\'e}todos computacionalmente eficientes para o 		processamento e interpreta{\c c}{\~ a}o de dados de campos potenciais}{Diego Takahashi}{Observat\'{o}rio Nacional, Brazil}{Em andamento - defesa prevista p/ 2021}{Orientador principal}
	
	\item\ThesisEntry{PhD}{\href{http://www.pinga-lab.org/thesis/kristoffer-phd.html}{Modelagem regional do campo de gravidade utilizando pontos de massa em coordenadas geod{\' e}sicas}}{Kristoffer A. T. Hallam}{Observat\'{o}rio Nacional, Brazil}{2019}{Orientador principal}
	
	\item\ThesisEntry{PhD}{\href{http://www.pinga-lab.org/thesis/siqueira-phd.html}{Otimiza\c{c}\~{a}o computacional do m\'{e}todo da camada equivalente}}{Fillipe C. L. Siqueira}{Observat\'{o}rio Nacional, Brazil}{2016}{Co-orientador}
	
\end{enumerate}

\subsubsection{\emph{Resumos em anais de congressos}}

\begin{enumerate}
	\item \bibentry{seg_2020_conv_eqlayer}
	\item \bibentry{sbgf_2019_gradgrav}
	\item \bibentry{seg_2019_gradgrav}
	\item \bibentry{eage_2017_fast_eqlayer}
	\item \bibentry{agu_fallmeeting_2016b}
\end{enumerate}

\subsubsection{\emph{Artigos em periódicos indexados}}

\begin{enumerate}
	\item \bibentry{takahashi_etal2020}
	\item \bibentry{siqueira_etal_2017}
	
\end{enumerate}


%%%%%%%%%%% Projeto do Leo Vital - doutorado
\subsection{Inversão de dados magnéticos para estimar a forma de corpos 3D (2016 -- presente)} \label{projeto-Leo}

\subsubsection{\emph{Objetivos}}

A interpretação de dados anomalias magnéticas na superfície da Terra é um importante desafio na área de geofísica de exploração devido a falta de unicidade dos problemas inversos que visam estimar as propriedades das fontes magnéticas. Está bem estabelecido na literatura que diferentes distribuições de magnetização em subsuperfície podem reproduzir, com igual precisão, as anomalias magnéticas medidas na superfície da Terra. Para contornar esta inerente ambiguidade, informação a priori deve ser introduzida para reduzir o conjunto de possíveis soluções compatíveis com as informações geofísicas/geológicas locais. O presente projeto propõe o desenvolvimento de métodos que presumem algum conhecimento sobre a distribuição de magnetização e estimam a forma das fontes magnéticas em subsuperfície por meio da solução de problemas inversos não-lineares. 

Parte deste projeto é uma colaboração com o Dr. Clive Foss, do \textit{Commonwealth
Scientific and Industrial Research Organisation} (CSIRO), na cidade de Sydney, Austrália, 
que é especialista em geofísica de exploração. O estudante de doutorado Leonardo B. Vital 
realizou estágio, sob a supervisão do Dr. Clive Foss, no âmbito do edital N$^{\circ}$ 
47/2017 PROGRAMA DE DOUTORADO SANDUÍCHE NO EXTERIOR 2017/2018 da CAPES, entre os anos 
de 2018 e 2019.

\subsubsection{\emph{Formação de recursos humanos}}

\begin{enumerate}
	
	\item\ThesisEntry{Doutorado}{Inversão radial de dados magnéticos para reconstrução de
		corpos 3D}{Leonardo B. Vital}{Observat\'{o}rio Nacional, Brazil}{Em andamento - defesa prevista p/ 2020}{Orientador principal}
	
	\item\ThesisEntry{Doutorado}{Métodos de inversão de dados magnéticos para estimar fontes regionais}{Marlon Cabrera Hidalgo-Gato}{Observat\'{o}rio Nacional, Brazil}{2019}{Co-orientador}
	
\end{enumerate}

\subsubsection{\emph{Resumos em anais de congressos}}

\begin{enumerate}
	\item \bibentry{seg_2019_radial_mag}
	\item \bibentry{aseg_2019_radial_mag}
\end{enumerate}


\subsubsection{\emph{Artigos em periódicos indexados}}

\begin{enumerate}
	\item \bibentry{hidalgo-gato_etal2020}
	
\end{enumerate}

%%%%%%%%%%% Projeto guarda-chuva - eqlayer mag
\subsection{Camada equivalente aplicada ao processamento e interpretação de dados magnéticos (2017 -- presente)} \label{projeto-guarda-chuva-eqlayer}

\subsubsection{\emph{Objetivos}}

Desenvolvimento de métodos para o processamento e interpretação de dados magnetométricos 
usando a técnica da camada equivalente. A princípio, os métodos estimarão a distribuição 
de intensidades de momento magnético sobre a camada de forma iterativa. Utilizando a 
distribuição de momentos magnéticos estimada, será possível realizar processamento 
(interpolação, cálculo de gradientes espaciais, continuação para cima/baixo) e 
interpretação (estimativa da direção de magnetização, definição dos limites horizontais 
das fontes) de dados magnéticos. 


\subsubsection{\emph{Formação de recursos humanos}}

\begin{enumerate}
	
	\item\ThesisEntry{PhD}{\href{http://www.pinga-lab.org/thesis/andre-phd.html}{Desenvolvimentos teóricos da camada equivalente e suas aplicações a dados magnéticos }}{André L. A. Reis}{Observat\'{o}rio Nacional, Brazil}{2019}{Orientador principal}
	
\end{enumerate}

\subsubsection{\emph{Financiamentos}}

\begin{enumerate}
	
	\item\FundingEntry{\href{http://cnpq.br/pagina-inicial}{Conselho Nacional de Desenvolvimento Cient{\' i}fico e Tecnol\'{o}gico (CNPq)}}{Camada equivalente aplicada ao processamento de dados magn{\' e}ticos}{308945/2017-4}{CNPq N$^{\circ}$ 12/2017 - Bolsas de Produtividade em Pesquisa - PQ}{39~600.00}{Mar/2018 -- Feb/2021}
	
\end{enumerate}

\subsubsection{\emph{Resumos em anais de congressos}}

\begin{enumerate}
	\item \bibentry{seg_2020_eqlayer_remanent}
	\item \bibentry{seg_2020_fast_mag_eqlayer}
	\item \bibentry{seg_2019_amplitude}
	\item \bibentry{sbgf_2019_amplitude}
	\item \bibentry{seg_2019_eql_magdir}
\end{enumerate}

\subsubsection{\emph{Artigos em periódicos indexados}}

\begin{enumerate}
	\item \bibentry{reis_etal2020}
	
\end{enumerate}


%%%%%%%%%%% Projeto com a França
\subsection{Caracterização magnética de feições estruturais em regiões de crosta oceânica próximas ao equador (2018 -- presente)} \label{projeto-baixas-latitudes}

\subsubsection{\emph{Objetivos}}

Desenvolvimento de métodos para a interpretação de anomalias magnéticos em regiões de 
baixa latitude. Este projeto é uma colaboração com a Dra. Marcia Maia e o Dr. Pascal 
Tarits, ambos do \textit{Institut Universitaire Européen de la Meer} (IUEM), França, 
que são especialistas em geofísica marinha e processos geodinâmicos. 
Entre os objetivos deste projeto está a interpretação dos dados adquiridos na expedição
COLMEIA, ocorrida em 2013, com o intuito de determinar da extensão das áreas com manto 
serpentinizado na região da Zona de Falhas Transformantes de São Paulo (ZFTSP). 
Os resultados obtidos até o momento possibilitaram investigar a distribuição de 
suscetibilidade magnética na elevação de Atobá. Para tanto, presumimos que a 
suscetibilidade magnética é isotrópica e que as fontes estão magnetizadas por indução.
A cobertura escassa de dados sobre a elevação de Atobá ainda não possibilitou 
estabelecer se há uma clara relação entre a distribuição de suscetibilidade e a 
região de manto serpentinizado. 



\subsubsection{\emph{Formação de recursos humanos}}

\begin{enumerate}
	
	\item\ThesisEntry{Mestrado}{\href{https://www-iuem.univ-brest.fr/lgo/soutenances-master-1-physique-marine/}{Modélisation numérique des anomalies magnétiques au niveau de la zone de fracture de Saint Paul}}{Line Colin}{Institut Universitaire Européen de la Meer (IUEM), França}{2019}{Co-orientador}
	
\end{enumerate}

\subsubsection{\emph{Financiamentos}}

\begin{enumerate}
	
	\item\FundingEntry{\href{http://www.faperj.br/}{Funda\c{c}\~{a}o Carlos Chagas Filho de Amparo \`{a} Pesquisa do Estado do Rio de Janeiro (FAPERJ)}}{Camada equivalente aplicada {\` a}  caracteriza{\c c}{\~ a}o magn{\' e}tica de fei{\c c}{\~ o}es estruturais em regi{\~ o}es de  crosta oce{\^ a}nica pr{\' o}ximas ao equador}{E-26/202.729/2018}{Jovem Cientista do Nosso Estado – JCNE/2018}{75~600.00}{Nov/2018 -- Out/2021}
	
\end{enumerate}


%%%%%%%%%%% Projeto RTP generalizada
\subsection{Interpretação de dados magnéticos produzidos por distribuições de magnetização heterogêneas (2020 -- presente)} \label{projeto-mag-heterogenea}

\subsubsection{\emph{Objetivos}}

Os objetivos gerais do presente projeto são (i) generalizar a redução ao 
polo para o caso em que as fontes magnéticas possuem direção de magnetização variada 
e (ii) desenvolver métodos que permitam aplicar a redução ao polo generalizada para 
interpretar dados magnéticos produzidos por amostras de rocha, corpos geológicos com 
potencial exploratório e fontes magnéticas crustais.
Este projeto é derivado daquele apresentado na subseção 
\ref{projeto-guarda-chuva-eqlayer} e tem relação com o que foi apresentado na subseção 
\ref{projeto-Andre}.


\section{Demais orientações e supervisões}

\begin{enumerate}
	
	\item\Student{Elder Yokoyama}{Pós-doutorado}{Supervisor}{Observatório Nacional}{2014}
	\item\Student{Kristoffer A. T. Hallam}{Pós-doutorado}{Supervisor}{Observatório Nacional}{2020}
	\item\Student{Andre D. Arelaro}{Mestrado}{Co-orientador}{Observatório Nacional}{Em andamento - defesa prevista p/ 2020}
	\item\Student{Shayane P. Gonzalez}{PhD}{Co-orientador}{Observatório Nacional}{Em andamento - defesa prevista p/ 2021}
	\item\Student{Larissa S. Piauilino}{PhD}{Co-orientador}{Observatório Nacional}{Em andamento - defesa prevista p/ 2022}
	\item\Student{Edson A. F. Luza}{Mestrado}{Orientador principal}{Observatório Nacional}{Em andamento - defesa prevista p/ 2022}
	\item\Student{Raimundo O. Sousa Jr.}{Mestrado}{Orientador principal}{Observatório Nacional}{Em andamento - defesa prevista p/ 2022}

\end{enumerate}

\section{Demais artigos em periódicos indexados}

\begin{enumerate}
	\item \bibentry{oliveirajr_etal2011}
	\item \bibentry{oliveirajr_barbosa2013}
	\item \bibentry{melo_etal2013}
	\item \bibentry{oliveirajr_etal_2013}
	\item \bibentry{uieda_etal2014}
	\item \bibentry{maurya_etal2020}
\end{enumerate}

\section{Demais resumos em anais de congressos}

\begin{enumerate}
	\item \bibentry{seg_2011}
	\item \bibentry{eage_2011}
	\item \bibentry{seg_2012}
	\item \bibentry{scipy_2013}
	\item \bibentry{eage_2014}
	\item \bibentry{sbgf_2019_4Dgrav}
\end{enumerate}

\section{Outras atividades}

\begin{itemize}
	\item Membro da comissão de pós-graduação em Geofísica \newline
	Observat\'{o}rio Nacional, Portaria ON 20/2020 - 5/08/2020, \textbf{2020 -- presente}

	\item Pesquisador visitante \newline
	Institut Universitaire Européen de la Meer (IUEM), França, \textbf{2018 -- 2019}

	\item Coordenador do programa de pós-graduação em Geofísica \newline
	Observat\'{o}rio Nacional, Portaria ON 22/2017 - 29/05/2017, \textbf{2017 -- 2018}

	\item Membro da comissão de dados abertos \newline
	Observat\'{o}rio Nacional, Portaria ON 7/2017 - 8/03/2017, \textbf{2017 -- 2019}
	
	\item Membro da comissão de pós-graduação em Geofísica \newline
	Observat\'{o}rio Nacional, Portaria ON 7/2015 - 18/03/2015, \textbf{2015 -- 2017}
	
	\item Responsável pela disciplina \textit{Métodos Computacionais Aplicados à Geofísica}
	\newline
	Observat\'{o}rio Nacional, \textbf{2015 -- presente}
	
	\item Responsável pela disciplina \textit{Métodos Potenciais}
	\newline
	Observat\'{o}rio Nacional, \textbf{2014 -- presente}
	
	\item Membro do corpo docente permanente do programa de pós-graduação em Geofísica
	\newline
	Observat\'{o}rio Nacional, \textbf{2014 -- presente}
	
	\item Membro da comissão do Programa de Capacitação Institucional (PCI) \newline
	Observat\'{o}rio Nacional, Portaria ON 44/2013 - 15/10/2013, \textbf{2013 -- 2014}
	
\end{itemize}